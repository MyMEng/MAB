%  
%  in CH1 give an real life example at the begining as a BACKGROOUND
%  describe bacis concepts
%  and go into more details? obvious
%  
%  
%  
%  in CH2 active learning or web optimization
%  
%  
%  
\documentclass[11pt, a4paper]{report}
%  notitlepage - abstract on the same page
\usepackage{indentfirst} % indent frst paragraph of section
\usepackage{fullpage} % full A4 page
\usepackage{amsmath}
\usepackage[pdftex]{graphicx}
\usepackage{cite} % BiTeX
\usepackage{lipsum}

\newcommand{\HRule}{\rule{\linewidth}{0.5mm}}

\begin{document}

\begin{titlepage}
\begin{center}
% Upper part of the page. The '~' is needed because \\
% only works if a paragraph has started.
\includegraphics[width=0.5\textwidth]{graphics/UOB-logo.png}~\\[4cm] % was 1cm

% \textsc{\LARGE University of Bristol}\\[1.5cm]

%\textsc{\Large Final year project}\\[0.5cm]

% Title
\HRule \\[0.4cm]
{ \huge \bfseries \emph{Multi-armed bandits} problem.\\
	Practical introduction to the problem for everyone.\\
 	Real life application. \\[0.4cm] }
\HRule \\[1.5cm]

% Author and supervisor
\begin{minipage}{0.4\textwidth}
\begin{flushleft} \large
\emph{Author:}\\
Kacper \textsc{\textbf{Sokol}}
\end{flushleft}
\end{minipage}
\begin{minipage}{0.4\textwidth}
\begin{flushright} \large
\emph{Supervisor:} \\
Dr.~David \textsc{\textbf{Leslie}}
\end{flushright}
\end{minipage}

\vfill

% Bottom of the page
{\large \today}
\end{center}
\end{titlepage}

% \title{\emph{Multi-armed bandits} problem.\\
% 	Practical introduction to the problem for everyone.\\
% 	Real life application.}
% \author{Kacper Sokol\\University of Bristol, UK}
% \date{\today}
% \maketitle
% \begin{flushright}
% Supervised by:\\
% \textbf{David Leslie}
% \end{flushright}
% \begin{center}
% \line(1,0){250}
% \end{center}

\begin{abstract}
This dissertation consists of two chapters. First one is a general introduction to theory underlying multi-armed bandits problem. Reader is assumed not to need any prior knowledge in the field, only basics of statistics and probability theory are required. Second chapter is a ...
\begin{center}
Keywords: \textbf{multi-armed, bandit, reinforcement, learning, ...}
\end{center}
\end{abstract}
\newpage
\tableofcontents
\newpage

\chapter{Theory behind multi-armed bandits}

\section{Background}

\section{Terminology}
Agent--- a person that decide what arm to pull based on $\tau$ function.
Multi-armed bandit($N$)--- a ``device'' with $N$ possible choices of action.~\cite{berry+firstedt}
Strategy ($\tau$)--- tells the player which arm to pull at given stage of game.
Game--- a sequence of arms pulled based on chosen strategy $\tau$.
Play--- a single arm pull at stage $m$ of the game.

\section{Assumptions}
\lipsum[1]

\section{Discount Sequence}
To specify the rules governing the ``significance'' of outcome from a single play at stage $m$ the \emph{discount sequence} is introduced. It is a vector of specified length, which can also be infinite to .






\chapter{Practical application of multi-armed bandits algorithms}

\bibliography{ref}{}
\bibliographystyle{plain}

\end{document}
